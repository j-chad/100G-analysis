\documentclass{scrartcl}
\usepackage[utf8]{inputenc}
\usepackage[activate={true,nocompatibility},final,tracking=true,kerning=true,spacing=true,factor=1100,stretch=10,shrink=10]{microtype}
\usepackage[english]{babel}
\usepackage{csquotes}
\usepackage{hyperref}
\usepackage[style=apa, autocite=inline]{biblatex}
\usepackage{siunitx}
\usepackage{todonotes}
\usepackage{parskip}
\usepackage{pgfplots}

\microtypecontext{spacing=nonfrench}
\pgfplotsset{compat=1.15}

\subject{BUSAN 100G: Digital Information Literacy}
\title{Piazza Visualisation \& Analysis}
\author{Jackson Chadfield}
\date{\today}

\begin{document}

\maketitle

\section{Topic}
I have decided to analyse the timing of Piazza posts relative to key dates.

\section{Source}

\section{Analysis \& Visualisation}

\section{Report Management}
This document was written in \LaTeX. Primarily through the online service Overleaf, but also occasionally compiled manually.

I use the distributed version control system git to keep versions of my source code available (this includes both this document and any code used). Since I am the sole contributor to this project and it is relatively small I used a simple linear master branch. I use GitHub as a centralised store for my repository. All versions of my source can be accessed online\footnote{\url{https://github.com/j-chad/piazza-analysis}}

I originally managed my references with Zotero, but switched to Mendeley as Zotero's automatically generated citation keys are unintelligible. Mendeley also has direct integration with overleaf. This means that as soon as I add a new source, it will automatically be included in my reference list. 

I use Bib-\LaTeX{} as my back-end for my referencing, which will automatically add a reference list and citations where needed.

This tool-set allows me to access my work from any location: either from overleaf, or by cloning my repository and manually compiling my document.

\end{document}
