\documentclass{scrartcl}
\usepackage[utf8]{inputenc}
\usepackage[activate={true,nocompatibility},final,tracking=true,kerning=true,spacing=true,factor=1100,stretch=10,shrink=10]{microtype}
\usepackage[english]{babel}
\usepackage{csquotes}
\usepackage{hyperref}
\usepackage[acronym]{glossaries}
\usepackage[style=ieee, autocite=inline, backend=biber]{biblatex}
\usepackage{siunitx}
\usepackage{todonotes}
\usepackage{parskip}
\usepackage{pgfplots}

\microtypecontext{spacing=nonfrench}
\pgfplotsset{compat=1.15}
\addbibresource{references.bib}

\makenoidxglossaries
\newacronym{api}{API}{Application Programming Interface}
\newacronym{http}{HTTP}{Hyper-Text Transfer Protocol}
\newacronym{json}{JSON}{JavaScript Object Notation}
\newacronym{csv}{CSV}{Comma Seperated Values}

\subject{BUSAN 100G: Digital Information Literacy}
\title{Piazza Visualisation \& Analysis}
\author{Jackson Chadfield}
\date{\today}

\begin{document}

\maketitle

\section{Topic}
I have decided to analyse the timing of Piazza posts relative to key dates.

\section{Source}
I used the Piazza internal \acrshort{api} to scrape data from their site. I was able to use an unofficial python wrapper\cite{piazza-api}, which allowed me to grab data without research into the structure of their \acrshort{http} \acrshort{api}.

I initially scraped the data into a \acrshort{json} file.

\section{Analysis \& Visualisation}

\section{Report Management}
This document was written in \LaTeX. Primarily through the online service Overleaf, but also occasionally compiled manually.

I use the distributed version control system git to keep versions of my source code available (this includes both this document and any code used). Since I am the sole contributor to this project and it is relatively small I used a simple linear master branch. I use GitHub as a centralised store for my repository. All versions of my source can be accessed online\footnote{\url{https://github.com/j-chad/piazza-analysis}}

I originally managed my references with Zotero, but found many problems with using it with Bib-\LaTeX{}. The citation keys it generated were messy and many fields were incorrectly populated. I tried Mendeley as an alternative but ultimately found it easier to write the database entries manually.

I use Bib-\LaTeX{} as my back-end for my referencing, which will automatically add a reference list and citations where needed.

This tool-set allows me to access my work from any location: either from overleaf, or by cloning my repository and manually compiling my document.

\printnoidxglossary[type=\acronymtype]
\printbibliography

\end{document}
