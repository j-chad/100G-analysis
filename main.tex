\documentclass{article}
\usepackage[utf8]{inputenc}
\usepackage[utf8]{inputenc}
\usepackage[activate={true,nocompatibility},final,tracking=true,kerning=true,spacing=true,factor=1100,stretch=10,shrink=10]{microtype}
\usepackage[english]{babel}
\usepackage{csquotes}
\usepackage{hyperref}
\usepackage[style=apa, autocite=inline]{biblatex}
\usepackage{siunitx}
\usepackage{todonotes}
\usepackage{parskip}

\microtypecontext{spacing=nonfrench}

\title{BUSAN 100G Project}
\author{Jackson Chadfield}
\date{\today}

\begin{document}

\maketitle

\section{Topic}
I have decided to analyse the timing of Piazza posts relative to key dates.

\section{Source}

\section{Analysis \& Visualisation}

\section{Report Management}
This document was written in \LaTeX. Primarily through the online service Overleaf, but also occasionally compiled manually.

I use the distributed version control system git to keep versions of my source code available (this includes both this document and any code used). Since I am the sole contributor to this project and it is relatively small I used a simple linear master branch. I use GitHub as a centralised store for my repository. All versions of my source can be accessed online\footnote{\url{https://github.com/j-chad/piazza-analysis}}

My referencing is managed with Zotero, which has the benefit of integration with Overleaf. This means that as soon as I add a new source through Zotero, it will automatically be included in my reference list. 

I use Bib-Latex as my back-end for my referencing, which will automatically add a reference list and citations where needed.

This tool-set allows me to access my work from any location: either from overleaf, or by cloning my repository and manually compiling my document.

\end{document}
